\documentclass[12pt]{article}
\usepackage{fancyhdr,fancybox,amssymb,epsfig,amsmath,latexsym,bm}
\usepackage{enumerate}
\usepackage{tikz}
\usepackage[all]{xy}
\usetikzlibrary{arrows}
\usetikzlibrary{decorations.markings}

\tikzset{big arrow/.style={decoration={markings,mark=at position 0.6
      with {\arrow[scale=2]{>}}}, postaction={decorate}, shorten >=1pt}}

% Code to make an augmented matrix (Usage: \begin{bmatrix}[ccc|c])
\makeatletter
\renewcommand*\env@matrix[1][*\c@MaxMatrixCols c]{%
  \hskip -\arraycolsep
  \let\@ifnextchar\new@ifnextchar
  \array{#1}}
\makeatother

\topmargin=-.5in
\headsep=0.2in
\oddsidemargin=0in
\textwidth=6.7in
\textheight=9.2in
\footskip=.5in

% Change this line to \def\bold{\mathbf} for ``upright boldface.''
\def\bold{\bm}

\newcommand{\F}{\mathbb{F}}
\newcommand{\N}{\mathbb{N}}
\newcommand{\R}{\mathbb{R}}
\newcommand{\Z}{\mathbb{Z}}
\def\0{\bold{0}}
\def\A{\bold{A}}
\def\B{\bold{B}}
\def\C{\bold{C}}
\def\I{\bold{I}}
\def\M{\bold{M}}
\def\P{\bold{P}}
\def\Q{\bold{Q}}
\def\a{\bold{a}}
\def\b{\bold{b}}
\def\c{\bold{c}}
\def\e{\bold{e}}
\def\p{\bold{p}}
\def\q{\bold{q}}
\def\t{\bold{t}}
\def\u{\bold{u}}
\def\v{\bold{v}}
\def\x{\bold{x}}
\def\<{\langle}
\def\>{\rangle}
\def\hat{\widehat}

\DeclareMathOperator{\Span}{Span}
\DeclareMathOperator{\rank}{rank}
\newcommand{\vv}[2]{\begin{bmatrix} #1 \\ #2 \end{bmatrix}}
\newcommand{\vvv}[3]{\begin{bmatrix} #1 \\ #2 \\ #3 \end{bmatrix}}

\newcommand{\ds}{\displaystyle}

\renewcommand{\footrulewidth}{1pt}
\newenvironment{proof}[1][Proof.]{\smallskip\begin{trivlist}
    \item[\hskip \labelsep {\sffamily #1}]}{\qed\end{trivlist}\bigskip}
\newenvironment{sol}[1][Solution.]{\smallskip\begin{trivlist}
    \item[\hskip \labelsep {\sffamily #1}]}{\qed\end{trivlist}\bigskip}
\newenvironment{fminipage}
{\setlength{\fboxsep}{15pt}\begin{Sbox}\begin{minipage}}
{\end{minipage}\end{Sbox}\fbox{\TheSbox}}
\newcommand{\qed}{\hfill \ensuremath{\Box}}
\newcommand{\disp}{\displaystyle}


\pagestyle{fancy} \lhead{{\sf Homework 4 $|$ Due February 9 (Monday); MCM participants exempt
}} \rhead{\thepage} \cfoot{{\sf Math 4500 $|$ Mathematical Modeling
    $|$ Spring 2015 $|$ M.~Macauley}}


\begin{document}

$\;$

\begin{enumerate}


  % Allman/Rhodes Exercise 3.3.2
\item Carry out the steps outlined below for the following
  predator--prey model:
  \begin{align*}
    P_{t+1}&=P_t(1+.8(1-P_t))-4P_tQ_t \\
    Q_{t+1}&=.9Q_t+2P_tQ_t
  \end{align*}
  \begin{enumerate}
  \item Compute the equilbria.
  \item Use MATLAB and the {\tt twopop} program to make an informed
    guess as to whether the equilbria are stable or unstable. Print
    out or sketch the phase portrait.
  \item Linearlize the model at each of the equilibria and compute
    eigenvalues to determine stability.
  \end{enumerate}



  % Allman/Rhodes Exercise 7.1.6 (modified)
\item One approach to preventing disease spread is to simply
  quarantine infectives. Suppose a disease is modeled by the SIR
  model, but people who get the disease are health-conscience and
  quarantine themselves. The net result is that a fraction $q$ of the
  infectives are prevented from having contacts with the
  susceptibles. Only $1-q$ of the infectives will be able to spread
  the disease.
  \begin{enumerate}
    \item Modify the equations of the SIR model to reflect this. What
      value of $q$ gives the usual SIR model?
    \item Quarantining can be viewed as a way of modifying the
      transmission coefficient. Suppose an SIR model has transmission
      coefficient $\alpha$, and a fraction $q$ of the infectives are
      successfully quarantined. Then the model with quarantining is
      identical to a standard SIR model with some other transmission
      coefficient $\alpha'$, the \emph{effective transmission
        coefficient}. Give a formula for $\alpha'$ in terms of
      $\alpha$ and $q$.
    \item Use the MATLAB program {\tt sir} to investigate the behavior
      of your quarantine model for $N=100$, $\alpha=0.001$, and
      $\gamma=0.05$, and vary $q$ from $0$ to $1$. Explain the
      qualitative behavior you see. Can you find a value of $q$ that
      prevents an epidemic from occurring, regardless of $I_0$?
      Estimate the smallest such $q$.
  \end{enumerate}



  % Allman/Rhodes Exercise 7.1.6
\item Another approach to preventing disease spread is vaccination of
  susceptibles. Suppose a public health organization offers a vaccine
  for a disease modeled by the SIR model. One simple model of this
  situation counts each successful vaccination in the removed class
  throughout the duration of the model. 
  \begin{enumerate}
  \item Suppose that all vaccinations occur before the time $t=0$. Even if
    this is not the case, we may assume it is -- why?
  \item Suppose with $N=100$, we have $I_0=1$, with the removed class
    composed of the fraction $q$ of the population that was
    successfully vaccinated. Give formulas for $S_0$ and $R_0$ (the
    initial number of recovered people, \emph{not} the basic reproductive
    number $\mathcal{R}_0$). What value of $q$ gives the usual SIR
    model?
  \item Repeat Part~(c) from the previous problem for this situation.
  \end{enumerate}



  % Allman/Rhodes Exercises 7.2.5
\item An isolated island population of $100$ individuals is exposed to
  a particularly deadly disease; an infected individual remains
  contagious until overcome by death after $4$ days. We want to
  predict the disease' effect on the community on a daily
  basis. Suppose initially one individual is stricken with the disease. 
  \begin{enumerate}
    \item What is the removal rate $\gamma$? 
    \item For what values of the \emph{relative removal rate}
      $\rho:=\gamma/\alpha$ will an epidemic occur? Use this to
      determine for what values of the transmission coefficient
      $\alpha$ an epidemic will occur.
    \item Use a computer program such as {\tt sir} to estimate the
      number of days until the epidemic peaks for the values of
      $\alpha=.003$, $.005$, $.01$, and $.0125$, presenting your data
      in a table. How does the magnitude of $\alpha$ relate to the
      time until the peak?
    \item Calculate the basic reproductive numbers and the relative
      remove rates $\rho$ for the values of $\alpha$ above, adding
      that information to your table.
  \end{enumerate}



  % Allman/Rhodes Exercises 7.3.4--8.
\item The following difference equation is called the \emph{SIS
  model}:
  \begin{align*}
    \Delta S&=-\alpha SI+\gamma I \\
    \Delta I&=\alpha SI-\gamma I\,.
  \end{align*}
  \begin{enumerate}
  \item What disease might be modeled well by the SIS framework?
  \item Use a computer program such as {\tt sir} or {\tt twopop} to
    explore the dynamics of the SIS model. Vary the parameters
    $\alpha$, $\gamma$, $N$, $S_0$, and $I_0$. Describe your findings.
  \item Solve for all equilibria $(S^*,I^*)$. Are these biologically
    reasonable? An equilibrium $I^*>0$ is called an \emph{endemic
      equilibrium}. Can an SIS disease be endemic?
  \item Since $S_t+I_t=N$ is constant, substitute $I_t=N-S_t$ back
    into the formula for $S_t$ and find a formula for $S_{t+1}$ in
    terms of $S_t$. Find a formula for $I_{t+1}$ in terms of $I_t$.
  \item For the SIR model, the threshold value $\rho:=\gamma/\alpha$,
    called the \emph{relative remove rate}, plays an important
    role. What does it represent? Is there an analogous threshold
    value for the SIS model? If so, find it. If not, explain why.
  \item For the SIR model, the basic reproductive number
    $\mathcal{R}_0$ plays an important role. How should one define
    $\mathcal{R}_0$ for the SIS model? Justify your answer. 
\end{enumerate}



\end{enumerate}

\end{document}

